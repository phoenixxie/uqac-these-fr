%!TEX root = these.tex
\begin{abstract}

Les technologies de l'information sont devenues une partie importante de notre vie. Bien que ces magnifiques techniques nous facilitent la vie et facilitent la vie, les accidents et les catastrophes causés par des dysfonctionnements de logiciels provoquent beaucoup de pertes de vies et de richesse qui peuvent en fait être évitées. La vérification et la validation de logiciels sont un ensemble de techniques visant à vérifier la fonctionnalité et à évaluer la qualité logicielle. La vérification de l'exécution est l'une des techniques couramment utilisées dans le domaine industriel. Son origine provient d'autres techniques de vérification, mais elle a aussi ses propres fonctionnalités et caractéristiques.

Le but de cette recherche est d'explorer les méthodes et les solutions pour améliorer deux aspects de la vérification de l'exécution: la collecte de données et l'évaluation de formules. Dans la première partie, nous présentons un canal de communication unidirectionnelle basé sur des codes optiques qui sont applicables pour la transmission de données dans un environnement spécifique. Ensuite, dans la deuxième partie, nous introduisons notre solution de l'évaluation hors ligne de logiques temporelles basées sur la manipulation bitmap et la compression bitmap. Les deux parties ont été écrites sous forme d'articles à publier, dont l'un a été publié, tandis que l'autre est en cours d'examen.

\end{abstract}

\keywords{vérification de l'exécution, logique temporelle linéaire, compression de bitmap, protocoles de communication optique, code QR.}