%!TEX root = these.tex
\begin{abstract}

Les technologies de l'information sont devenues une partie importante de notre vie. Bien que beaucoup de techniques magnifiques nous fassent vivre facile et confortable, les accidents et les catastrophes causées par des dysfonctionnements de logiciels conduisent beaucoup de pertes de vies et de la richesse qui peuvent en fait évitées. La vérification et la validation de logiciels est un ensemble de techniques visant à vérifier la fonctionnalité et à évaluer la qualité logicielle. La vérification à l'exécution est l'une des techniques couramment utilisées dans le domaine d'industrie. Elle a son origine à partir d'autres techniques de vérification, mais elle a aussi ses propres fonctionnalités et caractéristiques.

Le but de cette recherche est d'explorer les méthodes et les solutions pour améliorer deux aspects de la vérification à l'exécution: la collecte de données et formule évaluation. Dans la première partie, nous présentons un canal de communication unidirectionnelle basé sur les codes optiques qui sont applicables pour la transmission de données dans un certain environnement spécifique. Ensuite, dans la deuxième partie, nous introduisions notre solution de l'évaluation hors ligne de logiques temporelles basées sur la manipulation de bitmap et la compression bitmap. Les deux parties ont été écrites sous forme d'article à publier, dont l'un a été publié, tandis que l'autre est en cours d'examen.

\end{abstract}

\keywords{vérification à l'exécution, logique temporelle linéaire, compression de bitmap, protocoles de communication optique, code QR}