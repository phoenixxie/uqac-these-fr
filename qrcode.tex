%!TEX root = these.tex

\chapter{Communication en streaming et en temps réel avec les codes optiques}

Ce chapitre présente une version reformatée et traduite d'un article publié en 2016 dans IEEE Access, v.4, p.284-298 par K. Xie, S. Gaboury et S. Hallé.

%% -----------------------
%% Section: intro
%% -----------------------
\section{Introduction}\label{sec:qr:intro} %% {{{

La communication sans fil est une technologie qui permet à deux ou plusieurs pairs de communiquer sans câbles électriques ou conducteurs \citep{tse2005fundamentals} . Alors que la majorité de technologies de communication sans fil utilise les ondes radio comme leurs milieux de transmission, quelques autres utilisent la lumière, en particulier dans les situations où la technologie radio est difficile à fonctionner. La communication optique en elle-même remonte à l'utilisation de drapeaux, les signaux de fumée et les lampes signalétiques pour communiquer de l'information entre deux points avec l'utilisation d'un code spécifique \citep{burns2004}.

Récemment, une forme simple de communication optique, appelé \emph{Quick Response Code} (code QR) \citep{qrcode-about}, a émergé comme un raffinement de la technologie existante de code-barres unidimensionnels. En raison de leur exactitude et leur grande capacité, les codes QR ont été utilisés dans de nombreux domaines; les applications du traitement de tels codes ont également été portés à une variété de dispositifs, y compris les ordinateurs de bureau, les smartphones, et même les télévisions.

Cependant, jusqu'au présent les codes QR ont été utilisés pour la transmission de données statique. En général, un code est imprimée sur un milieu physique, tel qu'une feuille de papier, et il est lu par un appareil optique (généralement une caméra) pour le décodage à un moment ultérieur. Dans ce chapitre, nous explorons l'idée de l'expansion de codes QR et les transformons en un canal dynamique de communication unidirectionnelle. Dans un tel canal, un flux de données est transmis par une \emph{séquence de codes}; généralement, ces codes sont continuellement générés et affichés sur un appareil, et simultanément capturés et décodés par un autre, ce qui est similaire aux autres types de technologies de communication.

Après avoir brièvement décrit dans la Section~\ref{sec:qr:lecture} les bases de codes QR et d'autres technologies de communication sans fil, nous nous concentrons dans la Section~\ref{sec:qr:qrcode} sur le principe de communication de codes QR bruts. Particulièrement, nous essayons de trouver les limites intrinsèques d'un tel canal de communication, par l'analyse de l'influence de divers facteurs, tels que la densité de code, le nombre de codes affichés par seconde, etc. Les résultats d'un benchmark qui couvert plus de cent combinaisons différentes de paramètres permettent d'extraire les conditions optimales qui minimisent le taux d'erreur dans le décodage des images, tout en maximisant la quantité de données qui peuvent être transmises par unité de temps.

Ces premiers résultats indiquent que les flux de code QR peuvent en effet être utilisés comme un canal simple et unidirectionnel, mais que la communication sans erreur et la bande passante élevée sont plus ou moins impossible. Par conséquent, en tant qu'un deuxième temps, nous concevons un protocole qui est approprié pour la nature spécifique de communication de codes QR. Ce protocole, appelé BufferTannen, est décrit dans la section~\ref{sec:qr:protocol}. Il est capable d'encapsuler les données brutes, de fournit diverses capacités de signalisation, de pouvoir représenter les données semi-structurées (telles que JSON) sous une forme binaire compacte, et prend en charge le cadrage/décadrage et le streaming de données.

Une seconde expérience révèle la robustesse de ce schéma de transmission: en utilisant notre protocole spécialement conçu, le canal de communication créé par une personne qui pointe un smartphone à bout du bras vers un écran avec les code QR des codes QR vacillants, produit une bande passante suffisante pour transmettre l'audio à usage vocal en temps réel. La Section~\ref{sec:qr:experiments} présente l'environnement et les résultats des expériences de cette deuxième étape. Nous montrons aussi comment un morceau de données, coupé en plusieurs codes avec l'utilisation de BufferTannen, peut être reconstruit automatiquement par un utilisateur qui passe sa caméra sur une feuille de ces codes imprimés dans aucun ordre particulier.

Ce travail a été motivé par une application pratique dans le domaine de la vérification à l'exécution. Dans les recherches antérieures, nous avons proposé et officieusement expérimenté l'utilisation de codes optiques comme une forme de communication à couplage lâche entre un système logiciel et un moniteur externe qui reçoit les événements produits par ce système-là \citep{DBLP_conf/rv/LavoieLVGH14}. Dans ce contexte, la communication par les milieux optiques assure un isolement complet entre le système et son moniteur.

Bien que l'utilisation de séquences de codes QR a été officieusement suggéré dans le passé, au mieux de notre connaissance, notre recherche est la première enquête systématique du potentiel de codes QR pour envoyer les flux de données en temps réel.

La solution que nous proposons fournit une méthode de distribuer les données en streaming sans dispositifs dédiés de communication. Les appareils requis sont seulement une caméra commune (comme une webcam ou la caméra dans un téléphone portable) et une petite surface plane pour afficher les codes QR séquentielles --- par exemple, un écran d'ordinateur, une télévision ou un smartphone, ou même une feuille de papier.
%% }}} --- Section

%% -----------------------
%% Section: comment on lit les codes QR
%% -----------------------
\section{Communications sans fil}\label{sec:qr:reading} %% {{{

Cette section rappelle quelques technologies communes de communication sans fil. Malgré leur popularité et leur performance, chacun a ses propres limites et les scénarios d'application.

\subsection{Ondes Radio}

Le premier milieu évident de communication sans fil est grâce à l'utilisation d'ondes radio, dont le meilleur exemple est WiFi \citep{Comer:2008:CNI:1816918}, utilisé pour la mise en réseau sans fil de la zone locale. Ses variantes sont basées sur la famille de standards IEEE 802.11, et prennent en charge la mise en réseau centralisée (routage) et décentralisée (\textit{ad hoc}). Le Tableau \ref{tab:qr:wifi-protocol} montre les standards populaires de 802.11 et une partie de leurs spécifications.

\begin{table}[ht]
\begin{center}
\begin{tabular}{|c|c|c|c|}
\hline
Protocole &	Fréquence & Débit maximal de données physique	&	Portée intérieur \\
\hline
802.11a &	5 GHz &	54 Mbps &	35 m \\
\hline
802.11b &	2.4 GHz &	11 Mbps &	35 m\\
\hline
802.11g &	2.4 GHz &	54 Mbps &	38 m\\
\hline
802.11n &	2.4/5 GHz &	150 Mbps &	70 m\\
\hline
802.11ac &	5 GHz &	866.7 Mbps & 35 m\\
\hline
\end{tabular}
\caption{Résumé de protocoles WiFi \citep{theng2008ubiquitous,perahia2013next}}
\label{tab:qr:wifi-protocol}
\end{center}
\end{table}

Un deuxième concurrent dans cette famille est Bluetooth \citep{Comer:2008:CNI:1816918}, qui est utilisé à courte portée et généralement point à point entre les appareils. Sa portée varie d'environ 1 à 100 mètres en fonction de la classe d'énergie. Le Tableau \ref{tab:qr:bluetooth} montre différentes versions de Bluetooth et leurs débits de données spécifiques.

\begin{table}[ht]
\begin{center}
\begin{tabular}{|l|r|c|}
\hline
Version &	Débit de données	&	Portée \\
\hline
Version 1.2 &	1 Mbps &	\multirow{4}{*}{\pbox{20\textwidth}{Classe 1: 100 m; \\ Classe 2: 10 m; \\ Classe 3: 1 m}}\\
\cline{1-2}
Version 2.0 + EDR &	3 Mbps & \\
\cline{1-2}
Version 3.0 + HS &	24 Mbps & \\
\cline{1-2}
Version 4.0 &	24 Mbps & \\
\hline
\end{tabular}
\caption{Spécifications de Bluetooth \citep{gupta2013inside}}
\label{tab:qr:bluetooth}
\end{center}
\end{table}

Enfin, ZigBee \citep {farahani2011zigbee}, basé sur le standard IEEE 802.15.4, vise à mettre en œuvre une communication sans fil à courte portée avec une faible énergie et une batterie longue durée. Le Tableau \ref{tab:qr:zigbee} montre ses performances dans différentes bandes de fréquences.

\begin{table}[ht]
\begin{center}
\begin{tabular}{|l|r|c|}
\hline
Bande de fréquence & Débit de données	&	Portée \\
\hline
868--870 MHz &	20 kbps &	\multirow{3}{*}{\pbox{5 cm}{10--100 m, en fonction de la puissance de sortie et de l'environnement}}\\
\cline{1-2}
902--928 MHz &	40 kbps & \\
\cline{1-2}
2.4--2.4835 GHz  &	250 kbps & \\
\hline
\end{tabular}
\caption{Spécifications de ZigBee \citep{lee2007comparative}}
\label{tab:qr:zigbee}
\end{center}
\end{table}

Tous ces protocoles partagent un commun point: avant d'autoriser toute forme de communication entre deux extrémités, une certaine forme de \emph{découverte} ou \emph{configuration} de dispositifs est nécessaire. Ce processus est généralement achevé à travers l'établissement d'une \emph{connexion} avec états à long terme entre les extrémités.

\subsection{IrDA}

Dans une autre famille, on trouve les technologies utilisant les ondes infrarouges au lieu du signal radio \citep{sarkar2007ad}. Outre les longueurs d'onde différentes, ces technologies se détendre généralement les exigences de l'établissement d'une connexion, et permettent une communication plus ``on-the-fly'' entre deux appareils. En règle générale, une extrémité d'une liaison infrarouge attend les données entrantes, tandis que périodiquement, un autre appareil pointe au récepteur et émet les rayons infrarouges transformés à partir des petit morceaux de données sans nécessité de préavis.

Une importance particulière est le standard IrDA (Infrared Data Association); ses émetteurs envoient les impulsions infrarouges avec un angle de cône et une irradiance modérée alors que ses récepteurs peuvent être à moins d'un mètre ou plusieurs mètres de ceux-là, en fonction de l'énergie des émetteurs et de la position dans le cône. La communication IrDA est semi-duplex et fournit CRC de base. Le Tableau \ref{tab:qr:irda-rate} montre plusieurs schémas IrDA et leurs débits de données dans la portée spécifique.

\begin{table}[ht]
\begin{center}
\begin{tabular}{|l|c|c|}
\hline
Schéma & Débit de données & Portée \\
\hline
SIR &	2.4--115.2 kbps &	\multirow{4}{*}{\pbox{3 cm}{jusqu'à un mètre}} \\
\cline{1-2}
MIR &	0.576--1.152 Mbps & \\ 
\cline{1-2}
FIR &	4 Mbps & \\
\cline{1-2}
GigaIR &	512 Mbps--1 Gbps & \\
\hline
\end{tabular}
\caption{Débits de données de schémas de couche physique de IrDA \citep{millar1998irda}}
\label{tab:qr:irda-rate}
\end{center}
\end{table}

\subsection{Visible Light Communication}

Comme l'indique son nom, Visible Communication Light (VLC) \citep{komine2004fundamental} utilise les longueurs d'onde dans la gamme visible (400-700 nm) pour communiquer les données entre les pairs --- ceci est généralement réalisé par allumer et fermer rapidement une source de lumière, ce qui permet une forme de codage des données comme les codes Morse. Un récepteur (par exemple une cellule photo-électrique) pointé par la source de lumière détecte ce vacillement et le convertit en données numériques. Avec les lampes fluorescentes comme la source lumineuse, le débit de données peut atteindre 10 kbps, tandis qu'avec la technologie LED, le débit de données peut être aussi élevé que 500 Mbps. La gamme dépend surtout des spécifications différentes, mais parce que la lumière ne peut pas traverser les murs et peut également être affectée par le mauvais temps ou d'autres sources de lumière, sa portée et la fiabilité sont limitées~\citep{arnon2015visible}.

Ce mode de communication est intrinsèquement unidirectionnel, car on ne peut pas répondre à la source de lumière, reconnaître la réception de l'information, ou demander d'une retransmission en cas de la perte de données. Par conséquent, cette technologie est aussi celle qui nécessite le moins couplage entre un émetteur et un récepteur; Selon tous les moyens pratiques, l'émetteur n'a pas connaissance de la présence d'un récepteur, qui, de son côté, peut choisir de commencer à recevoir à tout moment. Nous verrons plus loin que cette caractéristique est également partagée avec le canal de communication de codes QR que nous essayons de concevoir.

%% }}} --- Section

%% -----------------------
%% Section: le protocole
%% -----------------------
\section{Flux de codes QR}\label{sec:qr:qrcode} %% {{{

Dans le court sondage précédent de technologies de communication sans fil sont mentionné les codes optiques, qui sont aussi un moyen de transport de données sans nécessité d'un milieu physique. Dans cette section, nous passons en revue le concept de codes QR, et discute l'idée de produire les flux de données par les séquences de tels codes.

\subsection{Aperçu de codes QR}

Un code QR (officiellement appelé ``Quick Response Code'') \citep{qrcode-about} est un code-barre à deux dimensions qui stocke des données, comme le montre la figure \ref{fig:qr:helloqr}. En comparaison avec le code-barre bien connu UPC, qui est linéaire (i.e.\ unidimensionnel), un code QR peut stocker plus d'informations dans une impression plus petit. \footnote{\url{http://www.qrcode.com/en/}} Le standard du code QR stipule que ces codes peuvent avoir une capacité aussi élevée que 7089 caractères numériques ou 2953 caractères à 8 bits, et qu'un code est représenté dans un tableau carré d'un maximum de 177 $\times$ 177 ``pixels'', appelé \emph{modules} \citep{iso18004}.

\begin{figure}
\centering
\includegraphics[width=1in]{helloqr.png}
\caption{Un code QR avec le texte ``Hello world!''}
\label{fig:qr:helloqr}
\end{figure}

\begin{table}[t]
\begin{center}
\begin{tabular}{|c|c|c|c|c|c|}
\hline
Niveau de & Bits de & Caractère  & Caractère  & Caractère & Kanji \\
correction d'erreur & données & numérique & alphanumérique & binaire & \\
\hline
L &	23,648 & 7,089 & 4,296 & 2,953 & 1,817\\
\hline
M & 18,672 & 5,596 & 3,391 & 2,331 & 1,435\\
\hline
Q & 13,328 & 3,993 & 2,420 & 1,663 & 1,024\\
\hline
H & 10,208 & 3,057 & 1,852 & 1,273 & 784\\
\hline
\end{tabular}
\caption[Capacités]{Capacités maximales de stockage de codes de la version 40}
\label{tab:qr:qrcode-capacity}
\end{center}
\end{table}

La capacité d'un code QR dépend principalement de son type de données, de sa version et de son niveau de correction d'erreur. Le type de données peut être \emph {uniquement numérique}, \emph{alphanumérique}, \emph{binaire}, ou \emph{Kanji}. La version de 1 à 40, détermine les dimensions d'un code, qui varient de 21 $\times$ 21 à 177 $\times$ 177 modules. Les codes QR utilisent une forme de codage à correction d'erreur qui peut choisir parmi quatre niveaux: \emph{Low} (L), \emph{Medium} (M), \emph{Quartile} (Q), et \emph{High} (H), comme il est indiqué dans le Tableau~\ref{tab:qr:qrcode-capacity}. Évidemment, comme le niveau augmente, plus de redondance est introduite dans le contenu du code, ce qui diminue sa capacité de stockage; Cependant, plus de données peuvent être restaurées si le code est sale ou endommagé. Avec les formes de détection de position inclus dans le symbole, un code QR peut être décodé en 360 degrés.

Avant de générer un code QR à partir d'un morceau de données, un générateur de code doit analyser les données d'entrée pour décider le mode et la version la plus efficace. Dans le codage de données, les caractères sont convertis en un flux de bits, et dans ce progrès, certains \emph{indicateurs de mode} et \emph{terminateurs} sont insérés pour les changements de mode. Le flux de bits est ensuite divisé en mots de code à 8-bits, et les caractères de remplissage sont nécessaires pour combler le nombre de mots de code de la version choisie. La séquence de mots de code générée est divisée en blocs selon le niveau de correction d'erreur spécifique et un mot de code de correction d'erreurs est généré pour chaque bloc. Ensuite, les mots de code de chaque bloc sont entrelacés et quelques bits restants sont ajoutés selon le besoin.

Dans l'étape suivante, le générateur met les modules de mots de code dans une matrice en noir et blanc avec la forme de recherche, les séparateurs, le forme de synchronisation et les forme d'alignement; il applique les formes de masquage, évalue et sélectionne ensuite la forme appropriée. Enfin, il génère le \emph{format} et \emph{l'information de version} et complète le code QR \citep{iso18004}.

Les étapes de décodage ne sont que tout simplement l'inverse de la procédure de codage. Dans un premier temps, le code QR doit être situé et les modules noirs et blancs sont reconnus comme 0s et 1s qui forment un tableau binaire. De ce tableau binaire, le décodeur reçoit l'information du format et de la version. Avec ces informations, il peut commencer à lire les caractères et les mots de code de correction d'erreurs, puis tente de détecter et de corriger les erreurs avec les mots de code de correction d'erreur selon le niveau de correction d'erreur approprié. Dans l'étape suivante, les mots de code de données sont divisés selon le \emph{indicateurs de mode} et les \emph{indicateurs de nombre de caractères}, et les caractères de données sont finalement décodés et sortis.

\subsection{Le cas du code QR Communication}

Le processus précité s'applique au codage au décodage d'un code unique contenant les données statiques. Nous enquêtons maintenant l'idée de l'utiliser les codes QR en tant qu'un canal de communication, où les données en temps réel seraient transformées en situation réelle comme une \emph{séquence} de codes QR, qui pourraient ensuite être optiquement capturés par un dispositif, et reconverti en flux de données d'origine à l'extrémité de réception.

L'utilisation de communication de codes QR présente plusieurs avantages dans une poignée de scénarios. Par exemple, la Marine américaine a enquêté l'utilisation de codes QR comme un ``sémaphore numérique''. La technologie proposée se concentre sur la détection de codes à basse résolution à partir de très longues distances, et souligne l'intérêt et les cas possibles d'utilisation de cette technologie dans un contexte militaire:

\begin{quote}
``Arguably the most significant advantage of QR code LOS [line of sight] communications is the fact that they can be conducted without emitting energy in the RF spectrum. In an emissions controlled (EMCON) environment, this will provide a critical ability to communicate between ships without increasing the possibility of position detection.'' \citep[p.\ 46]{richter-msc}
\end{quote}

Cependant, dans l'ouvrage cité, les codes sont considérés comme \emph{statique}, c'est-à-dire qu'ils ne changent pas au fil du temps pour former un flux de données, et plus ou moins agissent comme un substitut de drapeaux ou de signes. Néanmoins, l'absence de toute émission d'ondes radio dans la communication de codes QR s'avère un avantage attrayant dans certains scénarios.

Nous avons également vu dans la section précédente comment toutes les autres technologies, telles que Bluetooth ou IrDA, nécessitent un matériel dédié. En revanche, la communication de codes QR peut être réalisée à travers les codes imprimés sur une surface dure, ou par un dispositif capable d'afficher les images à une résolution suffisante: les écrans de télévision, les écrans d'ordinateur, les tablettes et les téléphones cellulaires. De même, la réception peut être faite par un appareil équipé d'une caméra commerciale normale. Cela peut convertir les appareils équipés de ce matériel courant en dispositifs de communication, même s'ils ne sont pas conçus à cet effet en premier lieu. On peut même imaginer les situations d'urgence dans lesquelles tous les moyens numériques de communication entre deux points ne fonctionnent pas. Si la ligne de vision peut être établie et un affichage et une caméra sont disponibles, l'utilisation de codes QR permet néanmoins de transmettre les données numériques --- sans doute beaucoup plus rapidement que le manuel écriture ou transcription.

Enfin, nous avons mentionné au début comment l'utilisation d'un canal de communication optique et strictement unidirectionnelle peut également être souhaitable, même dans les situations où la communication radio ou câble est disponible. Par exemple, dans le contexte de la vérification à l'exécution, l'exécution d'un système est actuellement observée par un processus externe appelé \emph{moniteur}. Pour empêcher le moniteur d'interférer avec l'exécution du système, il est souvent placé sur une machine séparée, avec un canal de communication qui transporte les événements d'une en faveur de l'autre. Cependant, dans les protocoles traditionnels tels que TCP, la nature bidirectionnelle d'une connexion présente un risque trop élevé d'attaques contre le programme de monitoring. En outre, certaines configurations de logiciels sont nécessaires pour brancher le moniteur au programme: les adresses IP, les noms de tube, les ports, etc., ce qui représentent trop de couplage dans de nombreux scénarios. Nous avons discuté dans le travail passé \citep{DBLP_conf /rv/LavoieLVGH14} comment l'utilisation d'un canal de communication optique peut atténuer ces problèmes en fournissant un plus grand isolement entre le système et son moniteur.

\subsection{Estimating Bandwidth and Error Rate}

However, one-way transmission introduces the possibility of losing frames during the process, due to the limitation of the physical devices or the vulnerability of the software. Moreover, it is impossible for the transmitter to be aware of any missing frames on the receiving end and resend them. Therefore, we need to analyze thoroughly this approach to estimate the \emph{recognition rate} and the \emph{transmission bandwidth} of such a communication channel.

The transmission of codes takes multiple parameters: each frame's data size, the number of generated frames per second (fps) and the error correction level. All three can have an important effect on the generation of QR codes and the resulting bandwidth. Larger data size leads to a higher symbol version of the QR code and more symbol modules, and with the same frame size, a higher correction level requires more symbol modules than a lower one.

Because the sender cannot be aware of any codes missed by the receiver, this one-way communication channel is actually a lossy channel, of which the \emph{effective} bandwidth can be calculated with the measured frame recognition rate. This represents the number of bits that are correctly received.
%
\begin{equation*}
  bandwidth = \mathit{fps} \times \mathit{frame\_bits} \times \mathit{recognition\_rate}
\end{equation*}

If the receiver finds that not all the frames are received, the only way is to make sure the sender sends all the frames again and again until the receiver gets all frames and stops; the actual bandwidth is:
%
\begin{equation*}
  bandwidth = \mathit{fps} \times \mathit{frame\_bits} \div \mathit{actual\_sent\_times}
\end{equation*}

The recognition rate is normally determined by the ability of the camera and the screen, the accuracy of the recognition algorithm and the code's complexity (i.e.\ the number of the displayed modules). However, within the ability of the camera and the screen, if we can send the same frame for more than once, meanwhile the value of \emph{fps} doesn't need to change, the practical recognition rate can be improved.

\begin{equation*}
\mathit{practical\_recognition\_rate} = 1 - (1 - \mathit{recognition\_rate})^{times}
\end{equation*}


%% }}} --- Section

%% -----------------------
%% Section: expériences
%% -----------------------
\section{Experiments}\label{sec:qr:experiments} %% {{{

In this section, we describe experiments in which we measure the accuracy of reading sequences of QR codes in various conditions. The purpose of these experiments is threefold:

\begin{enumerate}
\item assess whether data can be successfully transmitted through the reading of sequences of optical codes;
\item determine the parameters that maximize the decoding rate and bandwidth of the transmitted data;
\item from these results, determine the characteristics of a typical QR stream communication channel.
\end{enumerate}

\subsection{Experimental Setup}

Our set of experiments involves producing and displaying sequences of QR codes on one end, and capturing and decoding these sequences on the other. In our experimental environment, we used a Samsung 19-inch LED monitor as the transmitter and a high-definition Logitech webcam as the receiver. The camera was placed at a fixed distance of 50 cm from the screen. The resolution of the monitor was 1280$\times$1024 pixels. %The webcam can capture at most 30 frames per second with the highest resolution of 1920 pixels $\times$ 1080 pixels. Zooming was disabled. 
The camera was placed on a stable surface, with the optical code zone correctly in focus and covering the whole field of view. The computer used for the experiments is a laptop with the Intel Core i7-3632QM processor and 16 GB of memory. Figure \ref{fig:qr:setup} shows the setup used for the experiments.

\begin{figure}
\centering
\includegraphics[width=\linewidth]{expsetup.jpg}
\caption{Experimental setup for reading codes.}
\label{fig:qr:setup}
\end{figure}

In the development, we chose OpenCV\footnote{\url{http://opencv.org/}} to capture the images from the camera and ZXing\footnote{\url{https://github.com/zxing/zxing}} to generate and decode the QR codes. To reduce the CPU and memory's overhead of capturing and decoding, the captured images were transformed to 16-level grayscale. The data used to generate QR codes were randomly generated \emph{alphanumerics}. All benchmarking code is implemented in Java and is freely available.\footnote{\url{http://github.com/sylvainhalle/GyroGearloose}}

The code decoding depends on the quality and the complexity of the captured image. If the image is broken or blurred, it will be difficult to decode. And the algorithms of image recognition may have the probability of failure\citep{adel2006}. Therefore, our first step is to measure the ability of optical recognition libraries to properly read sequences of codes, irrespective of the actual data contained in these codes. Sequences of codes were generated by producing a character string of the form \verb+dddd#rrrr...+, where \verb+dddd+ is a sequential number starting from zero and incrementing by one on each successive code, and \verb+rrrr...+ is a random string of characters (different on each code) long enough to fill the code up to its maximum size. Each test consisted in filming the sequence of such codes and storing the sequential number of each correctly decoded image into a file. This allows us to determine the fraction of all codes that were correctly read; given the size of each code and the number of codes sent, this makes it possible to compute the bandwidth and decoding error rate.

Our experiments quickly tripped over what appears to be a bug in the ZXing image decoding library. When analyzing sequences of images captured by the camera to look for decoding errors, we discovered that a number of times, the decoding failed while the corresponding image seemed to have no apparent problem (no blurring, correct framing, etc.). Putting the offending codes back on screen and trying to properly decode them with the camera yielded no success, even after changing the code's size, the camera's position, lighting conditions, etc. This is all the more puzzling that codes immediately before and after the problematic one were correctly decoded in multiple frames, while being captured in the same conditions. Even sending the code's ``pure'' image directly back into the decoding algorithm, without going through a camera, produces a decoding error.

It therefore seems the library cannot recognize some of the codes it itself produces (Figure \ref{fig:qr:bad-code} shows such an example). This most probably indicates a bug in the library, which has persisted up to the latest version available at the time this chapter was written. Therefore, in the following, the reader should keep in mind that an unknown proportion of reading errors may be due to this purported bug, and not to the particular experimental conditions. This is the case, for example, for the gaps in the correction rate we shall observe in Figures \ref{img-exp1} and \ref{img-exp2}.

\begin{figure}
\centering
\includegraphics[width=.8\linewidth]{badqr.png}
\caption{A QR code generated by ZXing that ZXing itself cannot decode in the experiment.}
\label{fig:qr:bad-code}
\end{figure}

\subsection{Experimental Parameters}

The experiment seeks the combination of parameters that could maximize the bandwidth and minimize the error rate for the transmission of codes. The parameters that were considered are the following.

\subsubsection{Code Resolution}

The first parameter is code data size (i.e.\ the number of data bits contained in each code) and physical size (number of pixels used to display the code on screen). We varied the data size in increments of 500 bits, from 500 up to 4,500 bits. As shown in Table \ref{tab:qr:sample-sizes}, the largest QR code, which contains 4,500 bits of data using the highest error correction level, is 101$\times$101 modules large. We also fixed the code's physical size to 700$\times$700 pixels, which makes each module a square of at least 6$\times$6 pixels.

\begin{table}[ht]
\begin{center}
\begin{tabular}{llll}
Input data bits & Error correction level & Symbol version & Symbol size\\
\hline
\multirow{2}{*}{500} & L & 3 & 29$\times$29\\
& H & 5 & 37$\times$37\\
\hline
\multirow{2}{*}{1000} & L & 5 & 37$\times$37\\
& H & 9 & 53$\times$53\\
\hline
\multirow{2}{*}{1500} & L & 6 & 41$\times$41\\
& H & 11 & 61$\times$61\\
\hline
\multirow{2}{*}{2000} & L & 8 & 49$\times$49\\
& H & 13 & 69$\times$69\\
\hline
\multirow{2}{*}{2500} & L & 9 & 53$\times$53\\
& H & 15 & 77$\times$77\\
\hline
\multirow{2}{*}{3000} & L & 10 & 57$\times$57\\
& H & 17 & 85$\times$85\\
\hline
\multirow{2}{*}{3500} & L & 11 & 61$\times$61\\
& H & 18 & 89$\times$89\\
\hline
\multirow{2}{*}{4000} & L & 12 & 65$\times$65\\
& H & 20 & 97$\times$97\\
\hline
\multirow{2}{*}{4500} & L & 13 & 69$\times$69\\
& H & 21 & 101$\times$101\\
\hline
\multirow{2}{*}{5800} & L & 19 & 93$\times$93\\
& H & 30 & 137$\times$137\\
\hline\end{tabular}
\caption[Sample sizes]{Sample QR code sizes, according to their data size and error correction level \citep{iso18004}}
\label{tab:qr:sample-sizes}
\end{center}
\end{table}

\subsubsection{Code Rate}

The second experimental parameter we considered is the code rate, i.e.\ the number of codes displayed per unit of time. We initially selected 2, 4, 6, 8, and 10 codes per second (cps), and also considered up to 16 cps in a later phase of the experiment.

\subsubsection{Error Correction Level}

As we have seen, QR codes include additional data intended for error correction. We hence also varied the level of error correction used in each experiment, using either its highest setting (H) or its lowest (L).

\subsubsection{Camera Resolution and Rate}

The resolution of the camera was not considered as an experimental parameter. It was fixed to its maximal setting, 1920$\times$1080 pixels. Similarly, its frame rate was kept fixed at 30 frames per second. This corresponds to 1080p high-definition video, a setting expected to be found in the majority of recent and future video capture devices. We performed some informal tests with lower resolutions (down to 640$\times$480), which were globally conclusive, but did not deem relevant of including them in our detailed analysis.

\subsection{Experimental Results}

The product of all combinations of code size, error correction level and code rate produces a total of 90 different experiments. These experiments were repeated in three sets, differing in the way in which codes were displayed.

\subsubsection{Single Display}

In a first experiment, each code was displayed in sequence for a duration of $1/f$ second, where $f$ is the code rate. The bandwidth and decoding rate are shown in Figure \ref{img-exp1} for combinations of all parameters.

\begin{figure}
\begin{center}
\centering
\includegraphics[width=\linewidth]{data1.png}
\caption{Bandwidth and decoding rate in the first experiment}
\label{img-exp1}
\end{center}
\end{figure}

As one can see, the recognition rates of higher correction level were lower than the ones of lower correction level, with all other parameters being equal. This can be explained by the fact that the same amount of data, carried inside a code with a higher correction level, has to display more modules. For example, according to Table \ref{tab:qr:sample-sizes}, the modules of a 2,000-bit, H-level code are as small as those of a 4,500-bit, L-level code. Smaller modules, in turn, yields increased difficulty in recognition by the camera. Therefore, a first conclusion one can draw is that, surprisingly, effective bandwidth seems to be improved by using a \emph{lower} level of error correction.

With the same data sizes and correction levels, the figure shows that the recognition rate decreases as the code rate increases. This can be explained by the fact that, in a higher code rate, the same code occupies fewer camera frames, and hence has fewer chances of being correctly decoded in one of the frames. Moreover, the probability that a code change occurs at the moment a frame is taken (resulting in a blurry image showing part of two different codes) is also increased. In the L level, the decrease is slight, but in the H level, the decrease is dramatic when the code size reaches 3,000 bits. As the data size increases, the recognition rate drops constantly and considerably.

These figures seem to indicate that the ideal configuration for level L is 4,500 bits and 10 fps, which yields an effective bandwidth of 39.0 kbps; for level H, 4,000 bits and 10 fps result in a bandwidth of 24.6 kbps.

\subsubsection{Double Display}

Considering that the camera might have missed several frames, we performed a second experiment in which every QR code is displayed twice within a small time window. Hence, instead of displaying each code once for $1/f$ second, each code was interleaved with neighbouring codes and displayed twice for $1/2f$ second each time. This results in the same total exposure time for each code, but increases the diversity in the images captured by the camera.

\begin{figure}[ht]
\begin{center}
\centering
\includegraphics[width=\linewidth]{data2.png}
\caption{Bandwidth and decoding rate in the second experiment, where each code is displayed twice}
\label{img-exp2}
\end{center}
\end{figure}

The results are plotted in Figure \ref{img-exp2}. They show an increase in all recognition rates, which are now all higher than 90\%. This, in turn, increases the effective bandwidth; using the same settings as above, one can get a bandwidth of 43.0 kbps using level L, and 44.1 kbps using level H.

\subsubsection{Random Padding}

However, as we discussed earlier, not all QR codes are created equal; for the same resolution and error correction level, experimental results indicate that some codes seem to be more difficult to read than some others. Therefore, merely repeating the same image multiple times has no impact on that intrinsic ``hardness''. Our third experiment introduces yet another mechanism for boosting recognition rate.

This time, we tried to make the codes from the same input data different by appending, at the end of the data to be encoded, a small random string intended to change every time the code is to be displayed. Hence the same original data, if displayed twice, is prepended to a different random padding each time, yielding a slightly different array of bits. However, by virtue of the QR encoding schema, even a small change at the end of an array produces a completely different pattern of dots in the resulting QR code. Figure \ref{fig:qr:difcodes} shows an example of this phenomenon. Therefore, if a code is harder to read, the same data is also displayed in a largely different pattern of dots, increasing the odds of being properly picked up at least once.

\begin{figure}
\centering
\includegraphics[width=1in]{abcdefg.jpg}~\includegraphics[width=1in]{abcdeff.jpg}
\caption{Examples of two codes with slightly different data, but widely different dot patterns. The code on the left contains the string ``abcdefg'', while the one on the right contains ``abcdeff''.}
\label{fig:qr:difcodes}
\end{figure}

Although the objective reason for some codes being harder to read is unknown and out of the focus of this chapter, experimental results seem to confirm this hypothesis. We performed a third experiment where every input data was displayed three times with different generated QR codes. The recognition rate is better than before when the code rate is lower than 10 fps, as shown in Figure \ref{img-exp3}. % the best bandwidth in L level is 44.9 kbps and in H level is 43.4 kbps.

\begin{figure}[ht]
\begin{center}
\centering
\includegraphics[width=\linewidth]{data3.png}
\caption{Third experiment: triple display and random padding}
\label{img-exp3}
\end{center}
\end{figure}

These results led us to experiment with higher code rates; we added 12 cps, 14 cps and 16 cps. The codes were displayed twice. As the Figure \ref{img-exp4} shows, the maximum effective bandwidth in the result is 65.5 kbps using level L, and 68.3 kbps in level H level, using 16 cps and 4,500-bit codes.

\begin{figure}[ht]
\begin{center}
\centering
\includegraphics[width=\linewidth]{data4.png}
\caption{Fourth experiment: double display and higher code rates}
\label{img-exp4}
\end{center}
\end{figure}

\subsection{Partial Conclusions}

These initial experiments allow us to draw a few conclusions on the nature of a QR-based communication channel. First, although higher code rate and code size have a negative impact on the recognition ratio, the increased data that can be carried globally compensates for the higher error rate in terms of \emph{effective} bandwidth. Second, introducing repetition and varying the dot pattern for the same data increases the effective bandwidth; that is, showing two different codes for half the time is more effective than a single code for the same interval. Third, even for the smallest code sizes, the error rate of the channel is never zero, indicating that the channel is intrinsically lossy. 

From these findings, one can reasonably expect a QR code stream to provide a channel with an effective bandwidth of about 40 kbps, when displaying 10 4,000-bit codes per second using the random padding technique and L-level error correction. The decoding rate of the channel using these parameters should be of at least 95\%. Obviously, these findings apply to a fixed-camera setting. They do not take into account potential jitter, blurring or other effects that may occur in other contexts ---although an informal experiment described in Section \ref{subsub:swipe} tends to indicate the technology is relatively robust.

%% }}} --- Section

%% -----------------------
%% Section: le protocole
%% -----------------------
\section{A Protocol for One-Way, Lossy Communication Channels}\label{sec:qr:protocol} %% {{{

In this section, we propose an approach which uses continuous QR codes as a medium to achieve one-way data transmission.

\subsection{Design Goals}

In order to implement a communication channel, a specific protocol is essential; it should be well designed so that the data can be serialized and transferred without significant overhead. Besides, the protocol has to have the ability of splitting the to-be-transferred data into frames to generate the QR images. The result is BufferTannen, a Java software package dedicated to the serialization and transmission of structured data over limited communication channels.\footnote{\url{https://github.com/sylvainhalle/BufferTannen}} It provides a set of classes allowing the representation of structured data in a compact binary form. Contrarily to other systems, like Google's Protocol Buffers \footnote{\url{https://github.com/google/protobuf}}, defining new message types can be done at runtime and does not require compiling new classes to be used. Moreover, messages in BufferTannen cannot be encoded and decoded without prior knowledge of their structure. However, since messages do not contain information about their structure, they use much less space.

BufferTannen also defines a protocol allowing the transmission of messages.  Although any channel (TCP connection, etc.) can be used, BufferTannen was designed to operate on a channel with the following specifications, which are based on our initial experimental results:

\begin{itemize}
\item The channel is \emph{point-to-point}. The goal is to send information   directly from A to B; no addressing, routing, etc. is provided.
\item The channel is \emph{low-bandwidth} (that is, able to transmit a few hundred bytes at a time, possibly less than 10 times per second).
\item The channel is \emph{one-way}: typically, one side of the communication sends data that is to be picked up by some receiver. This entails that the receiver cannot acknowledge reception of data or ask the sender to transmit again, as in protocols like TCP.
\item The channel is \emph{lossy}. However, we assume that the channel provides a mechanism (such as some form of checksum) to detect when a piece of data is corrupted and discard it.
\item A receiver can start listening on the channel at any time, and be able to correctly receive messages from that point on. As such, the communication does not have a formal ``start'' that could be used, for example, to advertise parameters used for the exchange.
\end{itemize}

Therefore, the communication channel envisioned as the transmission medium for BufferTannen's messages can be likened in many ways to a slow broadcast signal, such as Hellschreiber \citep{hells}, slow-scan television \citep{slowtv}, Teletext \citep{teletext} or RBDS~\citep{rbds}.

BufferTannen's protocol aims at transmitting messages as reliably as possible under these conditions, while preserving the integrity of data and the ordering of messages. The low-bandwidth nature of the channel explains the emphasis on serializing messages in a compact binary form. Since the receiver cannot ask for any form of re-transmission, the protocol must provide for automatic re-transmissions of each message to maximize their chances of being picked up, while at the same time not confusing a re-transmission with a new message with identical content. Moreover, as the receiver can start listening at any moment, and that the schema of messages must be known in order to decode them, the schemas used in the communication must also be transmitted at periodic intervals.

\subsection{Schemas}
\setcounter{paragraph}{0}

The declaration of a data structure is called a \emph{schema}. Information can be represented in three different forms:

\begin{itemize}
\item Smallscii: A variable-length string of characters. Since BufferTannen is aimed towards limiting as much as possible the number of bits required to represent information, these strings are restricted to a subset of 63 ASCII characters (letters, digits and punctuation). Each character in a Smallscii string takes 6 bits, and each string ends with the 6-bit string \verb+000000+.

\item Integer: The only numerical type available in BufferTannen. When declared, integers are given a ``width'', i.e. the number of bits used to encode them. The width can be anything between 1 and 16 bits.

\item Enumeration: A list of predefined Smallscii constants. Enumerations can be used to further reduce the amount of space taken by a data element when its set of possible values is known in advance.
\end{itemize}

These basic building blocks can be used to write schemas by combining them using compound data structures:

\begin{itemize}
\item List: a variable-length sequence of elements, all of which must be of the same type (or schema). List elements are accessed by their index, starting
  with index 0.
\item FixedMap: a table that associates strings to values. The structure is fixed and the exact strings that can be used as keys must be declared. However, each key can be associated to a value of a different type.
\end{itemize}

These constructs can be mixed freely. The following represents the declaration of a complex message schema:

\begin{verbatim}
FixedMap {
  "title" : Smallscii,
  "price" : Integer(5),
  "chapters" : List [
     FixedMap {
       "name" : Smallscii,
       "length" : Integer(8),
       "type" : Enum {"normal", "appendix"}
     }
  ]
}
\end{verbatim}

The top-level structure for this message is a map (delimited by \verb+{+\dots\verb+}+). This map has three keys: \verb+title+, whose associated value is a Smallscii string, \verb+price+, whose associated value is a integer in the range 0-32 (i.e. 5 bits), and \verb+chapters+, whose value is not a primitive type, but is itself a list (delimited by \verb+[+...\verb+]+). Each element of this list is itself a map with three keys: a string \verb+name+, an integer \verb+length+, and \verb+type+ whose possible values are \verb+normal+ or \verb+appendix+.

Schemas can be represented in a compact and unequivocal binary representation as follows.

\paragraph{Integer} The declaration of an integer is encoded as the following sequence of bits:
%
\begin{verbatim}
ttt wwwww ddddd s
\end{verbatim}

The sequence \verb+ttt+ represents the element type, encoded on 3 bits. An integer contains the decimal value 6. The sequence of \verb+w+ indicates the integer's width in bits. The width itself is encoded over 5 bits. The sequence of \verb+d+ indicates the integer's width, in bits, when expressed as a delta value, i.e.\ as the difference with respect to an integer from a previous message. The width itself is encoded over 5 bits. The single bit \verb+s+ is the sign flag. If set to 0, the integer is unsigned; if set to 1, the integer is signed. Note that integers expressed as delta values are always encoded as signed integers; hence this flag only applies to integers occurring as full values.

\paragraph{Smallscii string} The declaration of a Smallscii string is simply coded as three bits representing the element type; a string contains the decimal value 2.

\paragraph{Enumeration} An enumeration must provide the list of all possible values it can take. It is formally represented as:
%
\begin{verbatim}
ttt llll [ssssss ssssss ... 000000 ...
 ssssss ssssss ... 000000]
\end{verbatim}

The element type is the decimal value 1, and the sequence \verb+llll+ is the number of elements in the enumeration, encoded on 4 bits. What follows is a concatenation of Smallscii strings defining the possible values for the enumeration. Each character is encoded on 6 bits, and the end of a string is signalled by the 6-bit sequence \verb+000000+.

\paragraph{List} The declaration of a list is as follows:
%
\begin{verbatim}
ttt llllllll ...
\end{verbatim}

The element type is the decimal value 3; the 8-bit sequence \verb+llllllll+ defines the maximum number of elements in the list. What follows is the declaration of the element type for elements of that list.

\paragraph{Fixed Map} The last element type is the fixed map, declared as follows:
%
\begin{verbatim}
ttt [ssssss ssssss ... 000000 ddd...]
\end{verbatim}

The element type is the decimal value 4; what follows is a Smallscii string defining the name of a key, followed by the declaration of the element type for that key; this is repeated for as many keys the map declares.

\subsection{Messages}
\setcounter{paragraph}{0}

A \emph{message} is an instance of a schema. For example, the following is a possible message abiding by the previous schema:

\begin{verbatim}
{
  "title" : "hello world",
  "price" : 21,
  "chapters" : [
    {
      "name" : "chapter 1",
      "length" : 3,
      "type" : "normal"
    },
    {
      "name" : "chapter 2",
      "length" : 7,
      "type" : "normal"
    },
    {
      "name" : "conclusion",
      "length" : 2,
      "type" : "appendix"
    }
  ]
}
\end{verbatim}

The reader familiar with JSON or similar notations will notice strong similarities between BufferTannen and these languages. As a matter of fact, elements of a message can be queried using a syntax similar to JavaScript. For example, assuming that \verb+m+ is an object representing the above message, fetching the length of the second chapter would be written as the expression:

\begin{verbatim}
m[chapters][1][length]
\end{verbatim}

This fetches the \verb+chapters+ value in the top-level structure (a list), then the second element of that list (index 1), and then the \verb+length+ value of the corresponding map element.

As with schemas, messages can be represented in a compact binary form.

\paragraph{Smallscii string} Strings are represented as a sequence of 6-bit characters, terminated by the end of string delimiter \verb+000000+.

\paragraph{Integer} Numbers are represented by the sequence of bits that encodes their value, without any terminating sequence: the number of bits to read is dictated by the size of the integer, as specified by the corresponding schema element. If the integer is signed, the first bit represents the sign (0 = positive, 1 = negative) and the remainder of the sequence represents the absolute value. %(Yes, this means that there are two ways of encoding 0 in a signed integer, either as -0 or as +0. Both will correctly be decoded as 0.)

\paragraph{Enumeration} An enumeration is simply made of the sequence bits corresponding to the appropriate value. Again, the number of bits to read is dictated by the size of the enumeration, as specified in the schema of the message to read. For example, if the enumeration defines 4 values, then 2 bits will be read. The numerical value $i$ corresponds to the $i$-th string declared in the enumeration.

\paragraph{List} A list begins by 8 bits recording the number of elements in the list. The remainder of the list is the concatenation of the binary representation of each list element. Since the type of each element and the number of such elements to read are both known, no delimiter is required between each element or at the end of the list.

\paragraph{Fixed Map} The contents of a fixed map is simply the concatenation of the binary representation of each map value. The key to which each value is associated, and the value type to read, are specified in the schema of the message to read, and are expected to appear exactly in the order they were declared. This spares us from repeating the map's keys in each message.

\subsection{Reading and Writing Messages}

In BufferTannen, both schemas and instances of schemas are represented by the same object, called \verb+SchemaElement+. An empty SchemaElement must first be instantiated using some schema; this can be done by either:

\begin{itemize}
\item Reading a character string formatted as above; or

\item Reading a binary string containing an encoding of the schema. As a matter of fact, in BufferTannen both messages \emph{and} schemas can be transmitted in binary form over a communication channel, and a method is provided to export the schema of some message into a sequence of bits.
\end{itemize}

Once an empty SchemaElement is obtained, it can be filled with data, again in two ways:

\begin{itemize}
\item By reading a character string formatted as above; or

\item By reading a binary string containing an encoding of the data.
\end{itemize}

Similar methods exist to operate in the opposite way, and to \emph{write} a message's schema or data contents either as a character string or as a binary string. This way, messages and schemas can be freely encoded/decoded using human-readable text strings or compact binary strings.

As one can see, for a message to be read or written, it is necessary first to instantiate an object with a schema. As a matter of fact, trying to decode a stream of data without first advertising the underlying schema will cause an error, even if the stream contains properly formatted data. Similarly, trying to read data that uses some schema with an object instantiated with another schema will also cause an error. In other words, no data can be read or written without knowledge of the proper schema to use.

This might seem restrictive, but it allows BufferTannen to heavily optimize the binary representation of messages. In the absence of a known schema, each message would require to carry, in addition to its actual data, information about its own structure. %Since a reader receiving a sequence of bits would not know in advance how to read it, special sequences would need to be added to notify the reader that what follows is a map with some number of keys, and then in turn each value for each key would also need to declare the structure of its own type, and so on.
%
Practically speaking, this amounts to repeating within each message the description of its schema, interspersed through the message data. On the contrary, if the schema is known, all this signaling information can be discarded: when receiving a sequence of bits, a reader that possesses the schema knows exactly how many bits to read, what data this represents and where to place it in the message structure being populated. This entails, however, that a receiver that does not know the schema to apply has no clue whatsoever on how to process a binary string.

To illustrate the interest of BufferTannen as a message encoding scheme, we consider the example of transmitting events from a video game to an external monitor.

\subsection{Segments}
\setcounter{paragraph}{0}

Messages and schemas are encapsulated into a structure called a \emph{segment}. A segment can be of four types:

\paragraph{Message segments} contain the binary representation of a message, along with a sequential number (used to preserve the ordering of messages received), as well as the number referring to the schema that must be used to decode the message. A message segment consists of a header structured as follows:

\begin{verbatim}
tt nnnnnnnnnnnn wwwwwwwwwwww ssss ...
\end{verbatim}

The header starts with two bits describing the type of the segment; a message segment contains the decimal value 1. The \verb+n+ and \verb+w+ sections describe the segment's sequential number and total length, both encoded on 12 bits. The four \verb+s+ bits provide the schema number in the schema bank that should be used to read this segment. The remainder of the segment is comprised of a map, list, Smallscii string or number, whose binary representation was described above.

\paragraph{Schema segments} contain the binary representation of a schema, which is associated to a number. Multiple schemas can be used in the same communication, hence creating a bank of schemas identified by their number. A schema segment consists of a header structured as follows:

\begin{verbatim}
tt nnnnnnnnnnnn ssss ...
\end{verbatim}

The header starts with two bits describing the type of the segment; a schema segment contains the decimal value 2. The \verb+n+ section describes the segment's sequential number, and the \verb+s+ section gives the the schema number in the schema bank this segment should be assigned to. The remainder of the segment is comprised of a binary string describing the schema, whose representation was described above.

\paragraph{Blob segments} are intended to carry raw binary data over the BufferTannen protocol.

\paragraph{Delta segments} contain the binary representation of a message, expressed as the difference (``delta'') between that message and a previous one used as a reference. Delta segments are used to further compress the representation of a message, in the case where messages don't change much over an interval of time.

\begin{verbatim}
tt nnnnnnnnnnnn wwwwwwwwwwww rrrrrrrrrrrr...
\end{verbatim}

The header starts with two bits describing the type of the segment; a delta segment contains the decimal value 1. The \verb+n+ and \verb+w+ sections describe the segment's sequential number and total length, both encoded on 12 bits. The \verb+r+ section gives the sequential number of another segment, relative to which the delta of the current segment are expressed. What follows is a binary string that describes the ``difference'' one must compute with respect to that segment to obtain the contents of the current one.

The computation of the delta is performed recursively on each element of the two messages to compare in the order they occur. It is defined for each element type as follows.

\begin{itemize}
\item Smallscii strings: if the corresponding strings are identical, emit the single bit \verb+0+. Otherwise, emit the bit \verb+1+ followed by the Smallscii string of the target message.
%
\item Integers: if the corresponding numbers are identical, emit the single bit \verb+0+. Otherwise, emit the bit \verb+1+ followed by the difference between the source and the target integer.
%
\item Enumerations: if the corresponding value of the enumerated type is the same, emit the single bit \verb+0+. Otherwise, emit the bit \verb+1+ followed by the integer value corresponding to the index of the value in the target message.
%
\item Lists: if both lists have the same elements in the same order, emit the single bit \verb+0+. Otherwise, emit the bit \verb+1+ followed by the binary representation of the target list.
%
\item Maps: recursively apply the previous rules for each key of the map.
\end{itemize}

One can see that delta segments apply only a coarse form of comparison. For example, no attempt is made to detect whether two lists differ by the addition or deletion of an element; the contents of the list is retransmitted in full whenever it is not identical to the original. Nevertheless, this technique allows substantial savings whenever a part of a data structure remains identical from one message to the next.

\subsection{Frames}
\setcounter{paragraph}{0}

The communication channel sends binary data in units called \emph{frames}. A frame is simply a set of concatenated segments in binary form, preceded by a header containing the version number of the protocol (currently ``1'') and the length (in bits) of the frame's content. Formally, the binary structure of a frame is as follows:

\begin{verbatim}
vvvv nnnnnnnnnnnnnn ffff... ffff...
\end{verbatim}

The \verb+v+ section consists of the 4-bit protocol version number, followed by 14 bits indicating the total length (in bits) of the frame. Each segment is appended directly to this 18-bit header. As each segment's header contains its own length, no further marshaling is required to correctly decode segment data.

When many segments are awaiting to be transmitted, the protocol tries to fit as many segments as possible (in sequential order) within the maximum size of a frame before sending it. This maximum size can be modified to fit the specifics of the communication channel that is being used. In the current incarnation of the protocol, segments cannot be fragmented across multiple frames. Hence a segment cannot exceed the maximum size of a frame.

Each frame is then converted into a QR code, with its binary content Base64-encoded as the code's text. This QR code can then be read at the receiving site, converted into a binary sequence, and parsed back into frames, segments and messages by applying the reverse transformations.

\subsection{Streaming Modes}
\setcounter{paragraph}{0}

BufferTannen is designed with two sending modes, respectively called ``Lake'' mode and ``Stream'' mode.

Lake mode is intended for the sending of a finite piece of data, such as a file, or a sequence of BufferTannen messages whose complete contents are known in advance. The data to be sent is divided into a finite set of segments, and the whole sequence of segments is repeatedly emitted through QR codes. If any frames are missed or incorrectly decoded, the infinite repetition of all segments makes possible to catch the missing data at the next loop. Ultimately, decoding errors may entail that the data needs to be read for more than one loop before it is completely received.

The use of Lake mode can be detected by frames carrying a non-zero value to their ``total segments'' header field. Hence a receiver that starts reading at any point through the sequence of frames knows how many segments in total are to be received, and the relative position of each segment in the data to be reconstructed. This makes Lake mode a relatively slow, but very robust optical data transmission scheme.

In Stream mode, the data is continuously read into segments which then form a stream of frames, and the frames are immediately sent. The reading process stops only when there is no more data to read. The frames already sent are removed from the memory, so there is no way to resend the data for several times. However, for the sake of the data consistency, we made a buffer for the clones of the sent frames, and after having sent a specific amount frames fetched from the original data, the frames in the buffer are resent again and then removed from the buffer. Therefore, Stream mode is intended to send realtime data, typically where where the data loss is acceptable and the consistency can be slightly sacrificed (e.g.\ audio or video).

\begin{figure*}
\centering
\begin{verbatim}
--------------------------------------------------------------------------
Sending mode:        lake
Buffer state:        [||>       ||:.::||:|||] 59% (130/219)
Progress:            0408/0000 (13.8 sec @30 fps)
Link quality:        22/30 [*******   ] (73%)  Global:   339/454 (74%)
Data stream index:   0
Resource ident.:     myfile.jpg
Processing rate:     35 ms/frame (27 fps)
--------------------------------------------------------------------------
\end{verbatim}
\caption{Part of the text interface of the QR code receiver operating in Lake mode}
\label{fig:qr:lake-gui}
\end{figure*}

Figure \ref{fig:qr:lake-gui} shows a portion of the text interface of our QR code receiver implementation. The interface shows that the frames being received are in Lake mode. The buffer state field indicates the progress of the reception. In the example, it shows that 130 out of 219 segments have been correctly received; the text bar at the left indicates to what portions of the total sequence these segments correspond. A section of the sequence that has not been received at all is indicated by a blank space; increasingly full portions of the sequence are represented respectively by the symbols \verb+.+, \verb+.+ and \verb+|+. The \verb+>+ symbol indicates the relative position of the last segment that was correctly read.

The ``Link quality'' field gives a realtime indication of the decoding rate. It shows that 22 of the last 30 images captured by the camera were correctly decoded, and that globally, 339 images were decoded out of 454 captured. The resource identifier and data stream index, carried by each frame, is also displayed.

\subsection{Experimental Results}
\setcounter{paragraph}{0}

The experiments of Section \ref{sec:qr:experiments} confirmed our intuition that optical code streams are an inherently unreliable and low-bandwidth communication channel. The compactness of the BufferTannen protocol can be motivated by an example from runtime verification. A particular video game, called Pingus, was instrumented to produce events containing the state of every character in the game. The schema for these events is shown in Figure \ref{fig:qr:pingus-schema}.

\subsubsection{File Transfer}

\begin{figure}
\centering
\begin{verbatim}
FixedMap {
  "pingus" : List [
    FixedMap [
      "id" : Integer(6),
      "x" : Integer(10),
      "y" : Integer(10),
      "velocity-x" : Integer(4),
      "velocity-y" : Integer(4),
      "state" : Enum {"floater",
         "basher", "builder",
         "athlete", "normal"}
    ]
  ]
}
\end{verbatim}
\caption{The schema of events produced by an instrumented video game.}
\label{fig:qr:pingus-schema}
\end{figure}

An event typically contains data for 50 characters, hence the map structure is repeated that many times. Sending such an event in clear-text format, without any whitespace, takes roughly 3,750 bytes. At a rate of 30 events per second, it takes 879 kbps of bandwidth to transmit the event stream. The same event in BufferTannen takes 1,856 \emph{bits}, or 232 bytes. This divides by more than 16 the bandwidth requirements for sending a stream of such events, yielding a bandwidth of 54 kbps.\footnote{Sending the same character string into Gzip shrinks it down to 716 bytes, which makes standard compression a less appealing alternative in that context.} From that point on, delta segments can be used to further reduce the stream's bandwidth, and transmit the remaining events using slightly more than 100 bytes each, consuming a bandwidth of approximately 24 kbps. Our previous experiments show that this is within the range of what one can reasonably expect to transmit using QR codes.

We then tested the ability of the BufferTannen protocol to mitigate these defects, through its use of repetition and its compact binary representation.\footnote{A video of BufferTannen in action is available online: \url{https://www.youtube.com/watch?v=GSL0md0TlY8}}

We chose to encode data into 4,000-bit codes, which, after the encapsulation of BufferTannen, amounts to a QR code of about 5,800 bits. With the combination of 5800 bits and L correction level, according to Table \ref{tab:qr:sample-sizes}, the symbol size is about 93 $\times$ 93 which is between the combinations of 3,500-bit, H-level and 4,000-bit, H-level. From the result of the last experiment, the combinations of 4,000-bit, H level and any sample fps has more than 95\% correction rate which is reliable. Secondly, the code rates were 4, 6, 8, 10, 12 fps. According to the last experiment, this configuration is reliable and supposed to be able to supply 23.2--69.6 kbps, and 16.0--48.0 kbps of bandwidth, respectively.
%
In the consideration of the practical application, we chose to transfer a sample file of which the size is 37,656 bytes, and we performed each experiment 20 times.

The result of the Lake mode experiment in Figure \ref{img-explake} shows that the best fps value is 10, and in this case the sample file needed to be transferred on average 2.4 times to make sure that the receiver could get all the frames. The average spent time is 17.27 seconds, from which the bandwidth is about 17.0 kbps.

\begin{figure}[ht]
\begin{center}
\centering
\includegraphics[width=\linewidth]{lake.png}
\caption{Time to send data in Lake mode}
\label{img-explake}
\end{center}
\end{figure}

In Stream mode, the percentage of received codes is important. In the experiment, according to Figure \ref{img-expstream}, the average completion ratios of all configurations are over 99\%, and the configuration of 12 fps needs approximately 13.11 seconds to send all frames with a data streaming channel of 22.4 kbps.

\begin{figure}[ht]
\begin{center}
\centering
\includegraphics[width=\linewidth]{stream.png}
\caption{Time to send data in Stream mode}
\label{img-expstream}
\end{center}
\end{figure}

\subsubsection{Paper Swiping}\label{subsub:swipe}

The ability to send streams of data in Lake mode can also be used to supplement QR codes' inherently limited capacity. When displaying a printed code on a piece of paper, large amounts of data can be carried only through increasing the code's resolution; however, resolution can only be increased up to a certain, predefined limit.\footnote{4,296 alpha-numeric characters, or 3,222 bytes assuming Base-64 encoding.} Moreover, for higher resolutions, the code may become hard to read using entry-level, low-resolution cameras. Therefore, it is safe to assume that, using existing QR code technology, no more than 3,000 bytes of data can be transferred using a QR code.

This limitation can be overcome by the use of the BufferTannen protocol. Although no code larger than roughly 4,000 bytes can be created, multiple such codes can be lined on a piece of paper. Each such code can be formatted to contain a single frame of data sent by the BufferTannen protocol in Lake Mode. It suffices for a user to swipe the camera over these multiple codes; by virtue of Lake Mode, the order in which the codes are scanned is irrelevant, and the complete piece of data can be correctly reconstructed from the individual frames. It is therefore possible to transmit theoretically unlimited amounts of data, while using codes of a lower resolution (this lower resolution being compensated by the presence of more than one code).

To verify this claim, we printed on a piece of paper the contents of a 37 kb file as a sequence of QR codes, processed as frames through BufferTannen in Lake Mode. We then hovered the camera over that sheet of paper at arm's length (see Figure \ref{fig:qr:paper-swipe}). The software's user interface displayed in real time the number of frames remaining to be decoded and their location in the complete stream, giving indications to the user as to which codes to swipe over. It shall be noted that the camera operated in ``film'' mode, and not in ``snapshot'' mode. In other words, images were continuously captured by the camera as it was being moved over the sheet; the user did not need to point and click at each individual QR code (which would be fastidious).

\begin{figure}
\centering
\includegraphics[width=\linewidth]{swipe.jpg}
\caption{Swiping the camera over a set of QR codes to reconstruct the contents of a larger file.}
\label{fig:qr:paper-swipe}
\end{figure}

The error correction level chosen was L and the sizes of raw data per code that we tested were 500, 750, 1,000, 1,250, 1,500, 1,750, and 2,000 bytes. With 500 bytes there were 76 codes while with 2,000 bytes there were only 19. After being encapsulated by BufferTannen protocol, the final sizes of data used to generate the QR codes became correspondingly 723, 1,055, 1,391, 1,724, 2,054, 2,389, and 2,722 bytes. The codes were printed at 300 dpi and 600 dpi on standard office paper, and we programmed to give each edge of every code 500 dots in the paper.

At 300 dpi, the codes of size ranging from 500 to 1,250 bytes were all decoded successfully and smoothly. However, when the size reaches 1,500 bytes, the decoding becomes more problematic. For some codes, we had to retry for several times by sticking the camera near the paper for a few seconds, yet among the total 31 codes, three remained impossible to decode at all. In the case of codes of 1,750 and 2,000 bytes, none of the codes could be decoded. This can be explained by the fact that higher density codes entail that each module of the code is smaller and hence harder to capture. For example, the edge of a 500-bytes and L-level code is about 77 modules, so each module can have approximately 6 dots in the paper, while the edge of a 2000-bytes, L-level code is 149 modules, so each module amounts to only 3 printed dots.

At 600 dpi, none of the codes could be decoded, no matter how many bytes they carried. The reason is that the codes printed at 600 dpi are too small and the camera has to approach very closely to the paper, but the captured images were all blurred and out of focus. It therefore seems that codes printed at 600 dpi are beyond the ability of a standard web camera.

Nevertheless, this experiment demonstrates the viability of the concept of hovering a camera across an array of QR codes. Our empirical findings indicate that a stream of data can be split into a set of QR codes of about 1,000 bytes each, the contents of which correspond to individual frames of the BufferTannen protocol containing the data to be transmitted. Swiping the camera over this set of codes, in no particular order, is sufficient to reconstruct on the device side the complete data contents.

%% }}} --- Section

%% -----------------------
%% Section: Conclusion
%% -----------------------
\section{Conclusion}\label{sec:qr:conclusion} %% {{{

In this chapter, we presented a solution for an one-way communication channel based on QR codes, and performed experiments to measure its performance. We first experimentally tested the characteristics of a QR data stream under various conditions, and extracted the parameters maximizing the effective bandwidth of the channel. However, since that channel is inherently error-prone and low-bandwidth, we then introduced BufferTannen, a protocol designed especially for this kind of channel. BufferTannen takes care of splitting, marshaling, and to some extend compressing the data to transmit in order to maximize the efficiency of the QR code stream. The feasibility of this approach was then empirically observed through a new set of experiments.

Within the limits of the protocol and of the communication channel, the presented results can be put to good use in a variety of situations. In limited environments where the use of radio signal or cables is forbidden or difficult, our approach can provide an easy way to communicate between peers, either as an emergency backup or as a primary means. Furthermore, the evolution of the quality of both display and image capture devices makes it possible to foresee increased transmission rates in the future. %as the performance of common devices keeps increasing, the symbols containing more data, or even some more complicated symbol formats can be brought into concern.
Finally, the aforementioned techniques could be turned into a bidirectional communication link in the case of endpoints equipped with both a camera and a display. In such a case, acknowledgments of correctly decoded images could be exchanged, which in turn would allow resending data on demand and increase the effective bandwidth.

%% }}} --- Section